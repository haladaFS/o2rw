%[HLAVICKA SOUBORU]
%Název článku = "Bernouliova rovnice"
%Zařazení = "Matematika"
%Předměty = "MKO", "MT"
%Zařazení = "zk", "zk"
%Autor = "Tomáš Druhý"


\documentclass[12pt,a4paper]{article}

\usepackage[utf8]{inputenc}
\usepackage[IL2]{fontenc}
\usepackage{mathtools}
\usepackage{gensymb}
\usepackage{siunitx}
\usepackage[czech]{babel}
\usepackage{graphicx}
\usepackage{amsmath}
\usepackage{booktabs}
\usepackage[left=2cm, right=2cm, top=1.5cm]{geometry}
\usepackage{indentfirst}
\usepackage{mathtools} % dalsi matematicke znaky, vyrazy
\usepackage{amsthm} % psani matematickych definic, vet a dukazu

\usepackage{xcolor}


%% citace:
\usepackage[backend=biber, style=numeric, sorting=nty, hyperref]{biblatex}
%\usepackage[backend=biber,style=numeric,sortcites,sorting=nty,backref,natbib,hyperref]{biblatex}
\addbibresource{unilbm.bib}
\usepackage{filecontents}

\usepackage{mathptmx}
\usepackage{anyfontsize}
\usepackage{t1enc}



\newcommand\ddfrac[2]{\frac{\displaystyle #1}{\displaystyle #2}}
\newcommand{\vect}[1]{\mathbf{#1}}

\textheight=240mm

\begin{document}
\thispagestyle{empty}

\noindent
\subsection*{Bernoulliho rovnice}

\subsubsection*{Testovací příspěvek}

\noindent
Rovnice, rovnice, rovnice.

\begin{equation}
	\frac{u_1^2 }{2} + \frac{p_1 }{\rho} + gh_1 = \text{konst.}
\end{equation}

\noindent
Testovací	rovnice, testovací text

\begin{equation}
\frac{\partial \vect{u}}{\partial t} + \nabla \cdot\vect{F}(\vect{u}) = 0
\end{equation}
\begin{equation}
	\frac{\vect{u}_i^{n+1}-\vect{u}_i^{n}}{\Delta t} = -\frac{\vect{F}_{i+\frac{1 }{2}}^{*}-\vect{F}_{i-\frac{1 }{2}}^{*}}{\Delta x} \quad ; \quad \vect{F}* = \frac{1}{2}\big[ \vect{F}(\vect{u}_L) + \vect{F}(\vect{u}_R) - \frac{\varepsilon }{f}(\vect{u}_R - \vect{u}_L)\big]
\end{equation}
%SPH:
%\begin{equation}
%	\frac{\text{d}\vect{u}}{\text{d} t} = \frac{\nabla\cdot\vect{F}(\vect{u})}{\rho}
%\end{equation}
%\begin{equation}
%	\frac{\text{d}\vect{u}_a}{\text{d} t} = - \sum_b m_b \bigg[\frac{\vect{F}_a }{\Omega_a \rho_a^2}\cdot \nabla_a W_{ab}(h_a) + \frac{\vect{F}_b }{\Omega_a \rho_a^2}\cdot \nabla_a W_{ab}(h_b) \bigg] - \sum_b \frac{m_b }{\bar{\rho}_{ab}}\bar{\varepsilon}(\vect{u}_a - \vect{u}_b)\frac{\vect{r}_{ab}}{\abs{\vect{r}_{ab}}}\cdot \bar{\nabla_a W_{ab}}
%\end{equation}

\subsubsection*{Smoothed particle hydrodynamics}


\noindent
Wendland kernel - funkce čtvrtého řádu:
$$
	W(r,h) = \alpha_d \bigg( 1 - \frac{q}{2} \bigg)^4 (2q +1); \quad 0 \leq q \leq 2
$$
kde $ \alpha_d $ je normovací konstanta s hodnotou $ 7/(4\pi h^2) $. \textcolor{blue}{Je možné použít celou řadu funkcí, které ve smyslu slabé limity konvergují k Dirackově delta distribuci.} Pořadujeme tedy

\begin{itemize}
	\item jednotkový integrál, konstanta $ \alpha_D $ tedy zajišťuje normování
	\item kompaktní nosič s tzv. \textit{cut-off} vzdálenosti $ \kappa h $
\end{itemize}

\end{document}
